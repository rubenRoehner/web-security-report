\newpage
\chapter{Executive Summary}
Im Rahmen der Vorlesung Web Security an der Hochschule Esslingen wurde ein Penetration Test auf 10 Systemen in einer Laborumgebung durchgeführt. Diese Systeme, darunter acht Webserver, ein DNS-Server und ein Client-System, befinden sich in den lokalen Netzen \url{10.0.72.0} und \url{10.0.68.0}. Im Rahmen des Penetrationstests wurden auf allen untersuchten Systemen schwerwiegende Sicherheitsprobleme festgestellt. Viele Sicherheitslücken basieren auf unzureichenden Passwortrichtlinien, fehlender oder schwacher Validierung von Benutzereingaben sowie veralteten und unsicheren Systemkomponenten.\\

\noindent Die Webanwendung \textit{Pediküre Salon Stolzfuß} weist mehrere Schwachstellen auf, darunter die Veröffentlichung von Systeminformationen, unzureichende Passwortrichtlinien und fehlende Überprüfung von Benutzereingaben. Dadurch ist es möglich, administrativen Zugriff auf den Webserver zu erlangen.

\noindent Die Webanwendung \textit{Grüner Drache} weist vor allem Probleme bei der Validierung von Benutzereingaben auf, wodurch administrativer Zugriff auf die Webanwendung und den gesamten Webserver möglich ist.

\noindent Die Webanwendung \textit{Sam's Gärtnerboard} weist Schwachstellen im Zusammenhang mit schwachen Passwortrichtlinien und fehlender Überprüfung von hochgeladenen Dateien auf. Dadurch ist es möglich, administrativen Zugriff auf das gesamte System zu erlangen.

\noindent Die Webanwendung \textit{Tuk Nextcloud} weist schwerwiegende Schwachstellen auf. Mangelnde Beschränkung der Zugriffsrechte, schwache Passwortrichtlinien und fehlende Validierung der hochgeladenen Dateien ermöglichen auch hier administrativen Zugriff auf das gesamte System.

\noindent Die Webanwendung \textit{Maggot's Pilzboard} weist Schwachstellen im verwendeten CMS-System auf, wodurch bekannte Sicherheitslücken ausgenutzt werden können, um administrativen Zugriff auf das gesamte System zu erlangen.

\noindent Die Webanwendung \textit{Auktionshaus Auenland} weist Schwachstellen beim Upload von Dateien auf, die unzureichend verifiziert werden und somit administrativen Zugriff auf das gesamte System ermöglichen können.

\noindent Die Webanwendung \textit{Tobold's Pfeifenkrautshop} weist Schwachstellen im verwendeten CMS-System auf, die durch Ausnutzung bekannter Sicherheitslücken zu administrativen Zugriffen auf das gesamte System führen können.

\noindent Das System \textit{Butterblume's Client} weist Schwachstellen in Form eines unischeren Synchronisationsskripts auf, das so modifiziert werden kann, dass administrativer Zugriff auf das gesamte System erlangt werden kann.

\noindent Die Webanwendung \textit{Zum Tänzelnden Pony} weist ebenfalls Schwachstellen in Form einer unzureichenden Überprüfung von Benutzereingaben und hochgeladenen Dateien auf. Dadurch ist auch hier ein administrativer Zugriff auf das gesamte System möglich.

\noindent Der DNS-Server \textit{Auenland DNS} verwendet schwache Anmeldedaten, die ausspioniert werden können und somit administrativen Zugriff auf das gesamte System ermöglichen.